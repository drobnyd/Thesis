\chapter{Databases}

A standalone database is not very useful as it is only some physical storage that never changes. To take the full advantage of it we need some means to define, create, maintain and control access to the database. That is purpose of a software called \definition{Database Management System (DBMS)}.

We already described why we want to use a database and roughly mentioned what are the pieces of data that we want to save there. 
Now let's take a look at what are differences between in database implementations and what to take in account when comparing database technologies.
That may be helpful when choosing the best suitable option for some specific data set to store or to see how storing of great amount of structured information can be approached. \\ 

The basic division of databases types is simple and binary - they are either Relational or Non-Relational.

There are Database Management Systems build around both, Relational Database Management System (RDBMS)

\subsubsection{Relational Databases}
A \definition{Relational Database} is a set of tables. A table consists of rows (also records) and columns. We can see such table as an object whose attributes are represented by columns and instances by rows. 
The important aspect is that relational tables tables carry both data that need to be stored by user and the relationships between the data as well. 
To store an atomic piece of data about instance a proper column is filled with a value.
Whereas to capture a relationship between objects the concept of keys is used. \\
A \definition{Key} is a subset of table's columns used for identifying a record. \\
A \definition{Primary Key} is a Key that non-ambiguously identifies a record in table and is used when referring to the record. \\
A \definition{Foreign Key} is a Key that uniquely identifies a record from a table (may be the same or a different one). \\
They are known also as SQL databases by the language - \definition{Structured Query Language (SQL)} which is used in RDBMS for managing data.

To be concrete, the most widely used relational database management systems are Oracle, MySQL, Microsoft SQL Server, PostgreSQL, IBM Db2 in this order. \footnote{The database technologies usage statistics are based on data from the most up to date version of website db-engines.com \cite{DatabaseEnginesStatistics19}.}

\subsubsection{Properties}
\cite{RelationalVsNonRelationalDatabases1} \cite{RelationalVsNonRelationalDatabases2}

\begin{itemize}
	\item Designed for managing structured data
	\item ACID compliance - database transactions are Atomic, Consistent, Isolated, Durable
	\item The technology is mature, well-established with large ecosystem and many developers have experiences with SQL and RDBMS
	\item Data integrity is enforced
\end{itemize}

\subsubsection{Non-Relational Databases}
A \definition{Non-Relational Database}, is any database that does not follow the relational paradigm. They are younger and were invented to overcome limitations that relational engines have. 
The ultimate aim is to be more effective when coping with Big Data - data that is fast growing and their structure may not be defined strictly (unstructured, semi-structured information) \cite{NonRelationalDBs}.
There are multiple ways that these requirements can be met so we will introduce more precise division \cite{DatabaseTypes}.
They are also commonly referred as NoSQL databases as the opposite of SQL databases. \\ 

In this work will focus only on the databases that are of the Relational kind. 

The main reason behind this is since NoSQL Databases have a flexible schema or are schema-less (there is no point in determining a database schema when data types of attributes or keys) modeling of these databases quite a new discipline and is hard to find an intersection among different approaches to NoSQL modeling.
Also concepts of higher abstraction models are omitted. \cite{NoSQLDatabaseModeling}

The fact to consider is that once a database is Relational we more or less know what to expect from it. The structure of these databases has a fixed skeleton. So a tool that would extract metadata from relational data models is potentially more powerful as it can be applied to more database technologies than a similar tool aimed for some specific type of Non-Relational database. 

Lastly, despite the Non-Relational may be growing in numbers and became a serious alternative, as it suits some use-cases better, the Relational still are, and in the near future will be, far more widely used.

\section{Means of Database Access}

Databases can be managed directly using Database Management Systems by a user who is using query language for accessing a database. 
However third party, or application, programs need also access the DBMS. 
In our work two types of programs will be connecting to databases when fetching metadata - modeling tools when undergoing a reverse-engineering process and Manta Flow in the extraction phase.
A solution is to provide them with an application programming interface (API) that provides a set of methods available in the programming language that the application program was written in, so it can use them.
Most commonly when the API is called its implementation translates the request so that to a specific DBMS driver that it is passed after understands it and performs the desired action. \\

A \definition{Connection String} is a textual information used to identify a data source and establish a connection with it. It it is made of pairs of keywords and values separated by semicolon, the keywords are parameters of the connection.

\paragraph{APIs to DBMS}
\begin{itemize}
	\item Open Database Connectivity (ODBC)\\
		General, language independent, ER/Studio, PowerDesigner
	\item Java Database Connectivity (JDBC)\\
		The Java ecosystem, Manta Flow, PowerDesigner
	\item ADO.NET\\
		.NET Framework
\end{itemize}